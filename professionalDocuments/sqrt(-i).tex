\documentclass[12pt]{article}

\usepackage{amsmath}

\author{Brian MacKie-Mason}
\title{sqrt(-i)}
\date{May 21, 2018}

\begin{document}

\maketitle

This is a simple demonstration of the beauty of math. One day I was interested in revisiting Euler's formula for complex numbers so I asked myself what the $\sqrt{-\imath}$ was equal to. Here are my findings:

\begin{equation}
	z = \sqrt{-\imath}
\end{equation}

Using Euler's formula (stated below) we can begin the simplification process.

\begin{equation}
	\exp{\imath \theta} = \cos{\theta} + \imath \sin{\theta}
\end{equation}

Let's start with $z = -\imath$. This is along the negative x-axis, and therefore we have $\theta = \frac{3 \pi}{2}$. To check:

\begin{equation}
	\exp{\imath \frac{3\pi}{2}} = \cos{\frac{3\pi}{2}} + \imath \sin{\frac{3\pi}{2}} = 0 - \imath
\end{equation}

Since we have the exponential representation of $\imath$ we can now take its square root, apply Euler's formula, and obtain a representation for $\sqrt{-\imath}$.

\begin{equation}
	\sqrt{\exp{\imath \frac{3\pi}{2}}} = \exp{\imath \frac{3\pi}{4}} = \cos{\frac{3\pi}{4}} + \imath \sin{\frac{3\pi}{4}} = \sqrt{\frac{1}{2}} \left( -1 + \imath \right)
\end{equation}

Therefore, $\sqrt{-\imath} = \sqrt{\frac{1}{2}} \left( -1 + \imath \right)$

\end{document}
